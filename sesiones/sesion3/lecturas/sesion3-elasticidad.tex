\documentclass[12pt,a4paper]{article}
\usepackage[utf8]{inputenc}
\usepackage{amsmath}
\usepackage{amsfonts}
\usepackage{amssymb}
\usepackage{geometry}

% Configuración de la página
\geometry{left=2.5cm,right=2.5cm,top=3cm,bottom=3cm}

\title{Capítulo 2: Elasticidad}
\author{Microeconomía}
\date{\today}

\begin{document}

\maketitle
\tableofcontents
\newpage

\section{Introducción}

En el campo de la microeconomía, el concepto de \textit{elasticidad} se utiliza para analizar en términos cuantitativos cómo se ajusta el mercado a las variaciones de los factores que influyen en la oferta y en la demanda. De esta manera, se profundiza el estudio de las relaciones de las variables que originan aumentos o caídas en las cantidades demandadas y ofertadas de los distintos bienes. Esta herramienta teórica permite conocer la amplitud o el grado de respuesta de una variable ante la modificación de otra variable de la cual depende. Por ejemplo, al subir el precio del queso, la cantidad demandada de queso baja; esta relación es enunciada por la \textit{Ley de demanda decreciente}; el cálculo de la \textit{elasticidad precio} de la demanda de queso nos permite saber en relación al aumento del precio y en \textit{cuánto} disminuirá la cantidad demandada de queso.

De esta manera, ampliamos nuestro análisis del funcionamiento del mercado al establecer los grados de relación entre algunas de las variables de la función de demanda, a saber el precio, el ingreso promedio, los precios de los bienes sustitutos y complementarios, con la cantidad demandada y la relación entre la variación del precio con las cantidades ofertadas. Por lo tanto, definiremos tres elasticidades por el lado de la demanda y una por el lado de la oferta:

\begin{enumerate}
    \item Elasticidad precio de la demanda
    \item Elasticidad ingreso de la demanda
    \item Elasticidad cruzada de la demanda
    \item Elasticidad precio de la oferta
\end{enumerate}

¿El efecto en el mercado de un aumento del 10\% en el precio de un chupetín será el mismo que el aumento del 10\% en el precio de un automóvil? ¿Tendrá el mismo peso sobre el gasto de un consumidor un incremento de la tarifa de luz eléctrica que la suba en el precio de las entradas al cine?

En este capítulo responderemos a estos interrogantes. Para ello, comenzaremos por comprender el aspecto matemático del indicador, elasticidad, para luego, una vez que hayamos incorporado la mecánica simple del cálculo, presentar para cada una de las definiciones algunos ejemplos de su aplicación.

\section{Elasticidad de la demanda}

En los análisis de la función de demanda, la cantidad demandada de un bien varía cuando:

\begin{itemize}
    \item Se modifica el precio del mismo.
    \item Se produce una modificación en el precio de bienes complementarios o sustitutos.
    \item Cambian los gustos y preferencias.
    \item Aumenta o disminuye el ingreso de los consumidores.
\end{itemize}

También afirmamos que en el primer caso, por tratarse de una variable endógena, se produce un movimiento a lo largo de la curva de demanda mientras que las demás situaciones provocan un desplazamiento de la curva de demanda, ya que nos referimos a variables exógenas.

En esta instancia, la propuesta es determinar qué tan sensible es la cantidad demandada de un bien a los cambios de las variables citadas recientemente en las viñetas. Para ello, analizaremos, primero, el comportamiento de la variable \textit{cantidad demandada} en relación con las variaciones del precio del bien. En segundo lugar, nos ocuparemos de determinar cómo responde la cantidad demandada frente a las variaciones del ingreso del consumidor. Y por último, veremos el grado de respuesta de la cantidad demandada a la modificación de los precios de otros bienes de la economía.

\subsection{Elasticidad precio de la demanda}

La elasticidad-precio de la demanda mide la magnitud o el grado de sensibilidad de la cantidad demandada de un bien frente a la variación del precio de ese bien. La fórmula para su cálculo es la siguiente:

\[
E_p = -\frac{\text{Variación porcentual de la cantidad demandada}}{\text{Variación porcentual del precio}}
\]

\subsubsection{Aspectos matemáticos}

\textbf{I. El signo negativo}

En la fórmula se observa el signo negativo que precede al cociente de las variaciones. ¿Por qué se lo incluye?

El precio y la cantidad demandada de un bien tienen una relación inversa, es decir que si el precio aumenta la cantidad disminuye y viceversa. Por esta razón, cuando relacionamos las variables precio-cantidad, el resultado será siempre negativo. Para transformar el valor de la elasticidad-precio en un número positivo –y así facilitar la interpretación de los resultados– se coloca el signo "menos" que precede a la fórmula. De esta manera, podemos determinar que la elasticidad-precio de un bien será siempre positiva.

\textbf{II. Variación porcentual de la cantidad demandada}

Al observar el numerador: "Variación porcentual de la cantidad demandada", nos preguntamos ¿qué es una variación porcentual?

Esta variación no la tenemos que pensar en términos de unidades, sino de un porcentaje de variación respecto de una cantidad inicial. Por ejemplo, si la cantidad demandada inicial es de 50 unidades y aumenta a 60 unidades, la variación porcentual será del 20\%. Tengamos en cuenta que, para la resolución de actividades, la información puede ser proporcionada como porcentaje, en cuyo caso este dato se incorpora al numerador de la fórmula directamente, o bien habrá que buscarlo a partir de los datos en términos de unidades. ¿Cómo se lo obtiene?

Primero encontramos la variación en unidades. La nueva cantidad menos la cantidad inicial:
\[
\Delta q = q_1 - q_0
\]

Luego, para obtener el porcentaje, dividimos la variación en unidades por la cantidad inicial y multiplicamos por cien:
\[
\frac{\Delta q}{q} = \frac{q_1 - q_0}{q_0} \times 100
\]

En nuestro ejemplo:
\[
\frac{\Delta q}{q} = \frac{60 - 50}{50} \times 100 = 0,2 \times 100 = 20\%
\]

\textbf{III. Variación porcentual del precio}

Se analizará el denominador de la fórmula: "Variación porcentual del precio". Análogamente, decimos que esta variación no la tenemos que pensar en términos de pesos, sino de un porcentaje de variación respecto de un precio inicial. Por ejemplo, si el precio inicial es de \$30 y disminuye a \$27, la variación porcentual será del –10\%. Recordemos que, a veces, la información puede ser proporcionada como porcentaje, en cuyo caso este dato se incorpora al denominador de la fórmula directamente, o bien habrá que buscarlo a partir de los datos brindados en términos de pesos. ¿Cómo se lo obtiene?

Primero hallamos la variación en pesos. El nuevo precio menos el precio inicial:
\[
\Delta p = p_1 - p_0
\]

Luego, para obtener el porcentaje, dividimos la variación en pesos por el precio inicial y multiplicamos por cien:
\[
\frac{\Delta p}{p} = \frac{p_1 - p_0}{p_0} \times 100
\]

En nuestro ejemplo reciente:
\[
\frac{\Delta p}{p} = \frac{27 - 30}{30} \times 100 = -0,1 \times 100 = -10\%
\]

\textbf{IV. Cálculo de la elasticidad-precio}

Ahora que se comprenden los componentes de la fórmula, buscaremos el valor de la elasticidad-precio.

\[
E_p = -\frac{\Delta q / q}{\Delta p / p} = -\frac{20\%}{-10\%} = 2
\]

Se tendrá en cuenta que no se trata de un porcentaje, ya que los signos se simplificaron en el cálculo, ni tampoco de un precio ni de una cantidad. Simplemente es el valor que toma la elasticidad-precio, que en nuestro ejemplo el resultado es 2.

\subsubsection{Interpretación de los resultados}

\textbf{I. Valores que puede tomar la elasticidad}

En primer lugar, nos preguntamos: ¿qué valores puede tomar? ¿Qué significa el valor obtenido?

Por definición, el valor de la elasticidad-precio es siempre positivo; puede ubicarse entre cero e infinito dependiendo de las variaciones de precios y de las cantidades consideradas:

\begin{itemize}
    \item \textbf{Si $E_p = 0$}: la demanda del bien es \textbf{perfectamente inelástica}. Esto quiere decir que frente a la modificación del precio (denominador de la fórmula), la cantidad demandada no reacciona ya que no se modifica el comportamiento del consumidor frente al cambio del precio y, por lo tanto, el numerador de la fórmula da 0.
    
    Se puede pensar en el caso de un medicamento único e imprescindible para la vida, al bajar o subir el precio la cantidad no se modificará, su consumo depende de lo necesario que es. Entonces:
    \[
    E_p = -\frac{0}{-10} = 0
    \]
    
    \item \textbf{Si $E_p < 1$}: es decir, un número entre 0 y 1, la demanda del bien es \textbf{inelástica}. Esto significa que frente a una variación del precio (denominador de la fórmula), la cantidad demandada se modifica, el consumidor reacciona al cambio de precio (movimiento a lo largo de la curva de demanda con pendiente negativa), pero su cambio es de una proporción menor al porcentaje de cambio del precio. La demanda es poco sensible a las variaciones del precio.
    
    Es el caso de la tarifa de gas, al subir un 5\% por ejemplo, la población reduce su consumo pero no podrá bajarlo mucho, sobre todo cuando es época invernal y solo lo hará en un 3\%; esto nos indica que se trata de un servicio inelástico, la variación de la cantidad demandada 3\% es menor a la variación del precio 5\%.
    
    Otro ejemplo numérico:
    \[
    E_p = -\frac{5\%}{-10\%} = 0,5
    \]
    
    El precio baja un 10\% y la cantidad demandada aumenta en un 5\%, o bien puede subir el precio en un 10\% y la cantidad demandada bajar en un 5\%:
    \[
    E_p = -\frac{-5\%}{10\%} = 0,5
    \]
    
    \item \textbf{Cuando $E_p = 1$}: la demanda de un bien tiene \textbf{elasticidad unitaria}. En este caso, frente a la modificación del precio el consumidor reacciona al nuevo precio, disminuyendo la cantidad que desea comprar (movimiento a lo largo de la curva de demanda), y esa modificación es de la misma proporción que la observada en el precio del bien.
    
    Este resultado es posible desde el punto de vista numérico pero es difícil poder encontrar de antemano una referencia a un bien en particular, ya que también es probable encontrarnos con resultados tales como 0,99 o 1,01 que pueden ser asimilables al resultado de una elasticidad unitaria. Un ejemplo donde numerador y denominador de la fórmula tienen el mismo valor absoluto sería:
    \[
    E_p = -\frac{10\%}{-10\%} = 1
    \]
    
    \item \textbf{Cuando $E_p > 1$}: la demanda del bien es \textbf{elástica}. Se dice que la demanda del bien es sensible a las variaciones del precio. Al cambiar el precio del bien, la cantidad demandada se modifica (movimiento a lo largo de la curva de demanda), pero este cambio es de una magnitud mayor a la variación producida en el precio. El numerador es mayor al denominador de la fórmula.
    
    Podemos analizar el caso del mercado de bicicletas. Si su precio baja en un 10\% la cantidad de bicicletas demandadas aumentará en más de un 10\%, por ejemplo en un 15\%.
    
    En este caso:
    \[
    E_p = -\frac{15\%}{-10\%} = 1,5
    \]
    
    Dentro de todos los valores de elasticidad-precio que son mayores a 1 existe un caso especial, cuando la $E_p$ tiende a un valor de infinito. En dicho caso, lo que muestra el resultado de la fórmula es que sin variación en el precio (0 en el denominador), la cantidad demandada reacciona (variación expresada en el numerador). Es el caso especial que identifica a una curva de demanda horizontal, es decir que a un determinado precio, los consumidores están dispuestos a adquirir cualquier cantidad del producto. A un precio distinto, no existe demanda.
    
    Ejemplo:
    \[
    E_p = -\frac{80\%}{0\%} = \infty
    \]
    Recordemos que si dividimos cualquier número por cero, el resultado tiende a infinito.
\end{itemize}

\textbf{II. Factores que determinan la elasticidad-precio}

¿Qué factores determinan el valor que toma la elasticidad-precio de la demanda de un bien en particular?

Para responder, necesitamos observar las numerosas fuerzas económicas, sociales y psicológicas que configuran el comportamiento de los consumidores. Es posible resumir los diversos factores en los siguientes puntos:

\begin{itemize}
    \item \textbf{Existencia de bienes sustitutivos cercanos al bien considerado.} Los bienes que tienen sustitutos cercanos tienden a tener una demanda más elástica, porque al subir su precio, el consumidor puede cambiarlo por el consumo de otro, cuyo precio permanece constante. Por ejemplo, las naranjas son fácilmente sustituibles por mandarinas.
    
    \item \textbf{Naturaleza de la necesidad que satisface el bien.} Los bienes que satisfacen una necesidad primaria tienden a tener una demanda inelástica. Por ejemplo, el precio de una visita al médico: aunque el precio baje el consumidor no hará muchas más visitas al médico.
    
    \item \textbf{Proporción de ingreso del consumidor gastada en el bien considerado.} Cuando el consumidor destina un porcentaje alto de su ingreso en el consumo del bien, la demanda tiende a ser más elástica. Cuando sube el precio del bien, también sube el porcentaje que ocupa sobre el total del ingreso, con el ingreso fijo, el consumidor tiene que disminuir mucho su cantidad demandada para compensar la suba del precio.
    
    Podemos, a modo de ejemplo, mencionar el valor del alquiler de una vivienda. Al aumentar el alquiler que se debe pagar, como esto gasto ocupa un porcentaje elevado del ingreso, supongamos entre un 40\% y 45\%, el peso del aumento obliga al consumidor a conseguir una vivienda que se alquile a un valor menor. De otro modo, tendrá que dejar de consumir una gran cantidad de otros bienes y servicios para compensar el aumento del alquiler.
    
    \item \textbf{Período de tiempo considerado.} Los bienes tienden a tener una demanda más elástica cuando el período de tiempo considerado es más largo. Cuando sube el precio del combustible, la cantidad demandada solo disminuye muy poco durante los primeros meses. A medida que pasa el tiempo, la gente compra autos que consumen menos combustible, opta por el transporte público, etc.
\end{itemize}

Estos son simplemente algunos de los factores condicionantes del valor que puede tomar la elasticidad-precio y que nos permiten comprender por qué existen bienes de demanda elástica y por qué existen bienes de demanda inelástica. Pero no debemos confundir estos condicionantes con los resultados de una clasificación de bienes.

Por ejemplo, en el punto 1.b. dijimos que un bien que satisface una necesidad primaria es probable que arroje como resultado del cálculo de elasticidad-precio un número entre cero y uno, es decir que la demanda sea inelástica; pero no es posible deducir, de manera inversa, que todos los bienes cuya demanda es inelástica sean bienes necesarios.

Por ejemplo, la demanda de pasajes de avión para llegar a Europa tiene un comportamiento inelástico, ya que no existen posibles formas alternativas de cruzar el océano Atlántico con rapidez y no por ello podemos decir que se trata de un bien necesario al igual que el consumo de leche.

\subsubsection{Algunas aplicaciones de la elasticidad precio de la demanda}

El concepto de elasticidad se utiliza para ampliar nuestra comprensión de la demanda y la oferta. Nos permite analizar no solo en qué sentido se ajustan las cantidades frente al cambio de los distintos factores que las afectan, sino también en cuanto tiempo lo hacen. A continuación, algunas de las aplicaciones más comunes del concepto elasticidad.

\paragraph{Efecto sobre el ingreso total de los empresarios}

Todas las empresas saben que la cantidad demandada se reducirá al subir el precio de venta. La información más importante se refiere a cómo se verá afectado el ingreso total de la empresa al variar el precio. El ingreso total (IT) recibido por el empresario será:

\[
\text{Ingreso total} = \text{precio} \times \text{cantidad demandada}
\]

El empresario está muy interesado en saber si su ingreso total va a aumentar o disminuir, y para ello necesita saber cómo reacciona la cantidad demandada frente a la variación del precio.

\textbf{Cuando la demanda es inelástica ($E_p < 1$):}

Al subir el precio, el ingreso total también sube, dado que la cantidad demandada disminuye en menor proporción a la suba del precio. Al disminuir el precio, el ingreso total también disminuye; la cantidad demandada aumenta en menor proporción a la baja del precio.

Veamos un ejemplo numérico: ¿Cuál es el ingreso total de un empresario que vende 1.000 unidades de un bien al precio de equilibrio \$5?

\[
IT = p \times q = 5 \times 1.000 = 5.000
\]

Si se produce un aumento en el precio de 20\%, por tratarse de un bien de demanda inelástica podemos asegurar que la disminución de la cantidad será proporcionalmente menor a ese 20\%, digamos por ejemplo que es del 5\%. (El resultado del cálculo de la elasticidad es de 0,25).

Observemos qué sucede con el ingreso total:

\[
p_1 = 5 + (5 \times 20\%) = \$6
\]
\[
q_1 = 1.000 - (1.000 \times 5\%) = 950 \text{ unidades}
\]
\[
IT = 6 \times 950 = 5.700
\]

\textbf{Cuando la demanda es elástica ($E_p > 1$):}

Al subir el precio, el ingreso total baja, ya que la cantidad disminuye en una proporción mayor a la suba del precio. Mientras que cuando baja el precio, el ingreso total aumenta, ya que la cantidad demandada aumenta en una proporción mayor a la baja del precio.

Clarifiquemos con un ejemplo numérico. ¿Cuál es el ingreso total de un empresario que vende 1.500 unidades de un bien al precio de equilibrio \$10?

\[
IT = p \times q = 10 \times 1.500 = 15.000
\]

Si se produce un aumento en el precio de 10\%, por tratarse de un bien de demanda elástica es posible asegurar que la disminución de la cantidad será proporcionalmente mayor a ese 10\%, digamos por ejemplo que es del 30\%. (El resultado del cálculo de la elasticidad es de 3).

Observemos qué sucede con el ingreso total:

\[
p_1 = 10 + (10 \times 10\%) = \$11
\]
\[
q_1 = 1.500 - (1.500 \times 30\%) = 1050 \text{ unidades}
\]
\[
IT = 11 \times 1050 = 11550
\]

En el caso especial de la elasticidad unitaria ($E_p = 1$), el ingreso total se mantiene constante, tanto cuando el precio sube como cuando el precio disminuye, dado que el ingreso total se maximiza cuando la elasticidad es igual a uno.

\paragraph{Efecto sobre el gasto total de los consumidores}

Así como analizamos el ingreso total del empresario a partir del valor que tome la elasticidad también es posible analizar cómo se modifica el valor del gasto total que tiene el consumidor.

\[
\text{Gasto total} = \text{precio} \times \text{cantidad demandada}
\]

\begin{itemize}
\item \textbf{Cuando la demanda es elástica ($E_p > 1$):} el gasto total del consumidor sube cuando el precio baja; la cantidad demandada aumenta en mayor proporción a la baja del precio. El gasto total baja cuando el precio sube, dado que la cantidad demandada baja en mayor proporción a la suba del precio.

\item \textbf{Cuando la demanda es inelástica ($E_p < 1$):} el gasto total del consumidor disminuye cuando el precio baja; la cantidad demandada sube pero en una proporción menor a la baja del precio. Cuando el precio aumenta el gasto total del consumidor sube, dado que la cantidad demandada disminuye pero en una proporción menor a la suba del precio.

\item \textbf{En el caso especial de la demanda con elasticidad unitaria ($E_p = 1$):} el gasto total se mantiene constante, tanto cuando el precio sube como cuando el precio disminuye.
\end{itemize}

\section{Elasticidad ingreso (o Elasticidad renta)}

La \textit{elasticidad ingreso} mide la magnitud o el grado de sensibilidad de la cantidad demandada de un bien frente a la variación del ingreso de los consumidores:

\[
E_y = \frac{\text{Variación porcentual de la cantidad demandada}}{\text{Variación porcentual del ingreso}}
\]

\subsection{Aspectos matemáticos}

\textbf{I.} Al igual que para la elasticidad-precio de la demanda, el numerador es "Variación porcentual de la cantidad demandada", por lo que para obtener dicho valor, se procede de la misma manera detallada en el apartado II del punto I.a:

\[
\frac{\Delta q}{q} = \frac{q_1 - q_0}{q_0} \times 100
\]

\textbf{II.} En el denominador de la fórmula tenemos: "Variación porcentual del ingreso". Como en los casos anteriores, el cálculo de esta variación no se realiza en términos de pesos, sino de un porcentaje de variación respecto de un nivel inicial de ingreso. Por ejemplo, si el ingreso inicial es de \$1.000 y aumenta a \$1.200, la variación porcentual será del 20\%. Como ya dijimos, para la resolución de actividades, la información puede ser proporcionada como porcentaje, en cuyo caso este dato se incorpora al denominador de la fórmula directamente, o bien habrá que buscarlo a partir de los datos en términos de pesos. ¿Cómo lo obtenemos?

En primer lugar, hallamos la variación en pesos. El nuevo ingreso menos el ingreso inicial:

\[
\Delta Y = Y_1 - Y_0
\]

Luego, para obtener el porcentaje, dividimos la variación en pesos por el ingreso inicial y multiplicamos por cien:

\[
\frac{\Delta Y}{Y} = \frac{Y_1 - Y_0}{Y_0} \times 100
\]

En nuestro ejemplo:

\[
\frac{\Delta Y}{Y} = \frac{1200 - 1000}{1000} \times 100 = 0,2 \times 100 = 20\%
\]

\textbf{III.} Para obtener el valor de la elasticidad ingreso aplicamos la siguiente fórmula:

\[
E_y = \frac{\Delta q / q}{\Delta Y / Y}
\]

Para este ejemplo, necesitamos conocer la variación porcentual de la cantidad. Supongamos que la cantidad demandada aumentó un 5\%, entonces tenemos:

\[
E_y = \frac{5\%}{20\%} = 0,25
\]

\textbf{Importante:} Para la elasticidad ingreso no podemos presuponer una relación de tipo directa o inversa en el comportamiento de las variables. En consecuencia, debemos considerar los signos de las variaciones, tanto del numerador como del denominador, y efectuar el cálculo tal como se presenta. Por lo tanto, el resultado será positivo si ambas variables aumentan o si ambas disminuyen y será negativo si se comportan en forma opuesta.

\subsection{Interpretación de los resultados}

¿Qué valores puede tomar la elasticidad ingreso? ¿Qué significa el valor obtenido? De acuerdo con lo expuesto en el párrafo anterior destacado, la elasticidad ingreso puede ser mayor o menor que cero. Analicemos qué representan tales valores:

\begin{itemize}
\item La elasticidad ingreso es mayor que cero ($E_y > 0$) cuando la relación entre las variables es directa; es decir que si el ingreso sube, la cantidad demandada del bien aumenta y si el ingreso baja, la cantidad demandada disminuye. Este es el comportamiento que podemos percibir frente a la mayoría de los bienes (si tenemos más dinero compramos más y si tenemos menos dinero compramos menos), es por eso que los denominamos \textit{bienes normales}.

\item Si por el contrario, la cantidad demandada de un bien disminuye cuando el ingreso sube o aumenta cuando el ingreso baja, el resultado de la elasticidad ingreso será menor que cero ($E_y < 0$). Cabe preguntarse, ¿existen bienes que desestimamos cuando aumenta nuestro ingreso? Si nos detenemos a pensar en nuestro comportamiento individual, encontraremos varios ejemplos: artículos de segundas marcas, de baja calidad, alimentos básicos, etc. Esos mismos bienes son a los que recurrimos cuando la situación económica nos apremia. Por este motivo, denominamos \textit{bienes inferiores} a aquellos que tienen una elasticidad ingreso menor a cero.

\item Ahora bien, dentro del rango de valores posibles mayores a cero ($E_y > 0$), es decir para los \textit{bienes normales}, podemos hacer una subclasificación: los que toman valores menores a uno ($0 < E_y < 1$) que denominaremos \textit{bienes necesarios} y aquellos cuya elasticidad ingreso es mayor que uno ($E_y > 1$), llamados \textit{bienes de lujo}. Si bien, tanto los bienes necesarios como los de lujo son bienes normales lo que caracteriza a los primeros es que, si el ingreso aumenta en un 20\%, por ejemplo, la cantidad demandada del bien aumentará pero en menor proporción, digamos por ejemplo un 10\%. En cambio, para los bienes de lujo el aumento de la cantidad demandada del bien será proporcionalmente mayor al aumento del ingreso, por ejemplo de 30\%.
\end{itemize}

\subsection{Tipo de bienes}

\begin{itemize}
\item \textbf{Bien normal}: aquel cuya demanda aumenta al aumentar el ingreso de la población.
\item \textbf{Bien inferior}: aquel cuya demanda disminuye al aumentar el ingreso de la población.
\item \textbf{Bien necesario}: es un bien normal, cuya demanda aumenta en una proporción menor al aumento del ingreso.
\item \textbf{Bien de lujo}: es un bien normal, cuya demanda aumenta en una proporción mayor al aumento del ingreso. Debe entenderse como opuesto a necesario [bienes de los cuales podemos prescindir] y no como sinónimo de suntuoso o valioso. Es así, que podríamos considerar dentro de esta categoría una diversidad de bienes (y servicios) tales como revistas, vinos, entradas de espectáculos, bombones, etc.
\end{itemize}

\section{Elasticidad cruzada}

La elasticidad cruzada ($E_c$) define la relación entre dos bienes que se manifiesta a través de la respuesta de la variable cantidad demandada de un bien frente a la variación del precio de otro bien.

\[
E_c = \frac{\text{Variación porcentual de la cantidad demandada del bien A}}{\text{Variación porcentual del precio del bien B}}
\]

\subsection{Aspectos matemáticos}

\textbf{I.} En el numerador tenemos: "Variación porcentual de la cantidad demandada del bien A"

Al igual que en los casos anteriores, la calculamos obteniendo el porcentaje de variación respecto de una cantidad inicial de dicho bien: primero hallamos la variación en unidades, la nueva cantidad menos la cantidad inicial, luego la dividimos por la cantidad inicial y multiplicamos por cien:

\[
\frac{\Delta q_A}{q_A} = \frac{q_{A1} - q_{A0}}{q_{A0}} \times 100
\]

\textbf{II.} Consideremos ahora el denominador de la fórmula: "Variación porcentual del precio del bien B".

Para el cálculo, primero obtenemos la variación en pesos, el nuevo precio del bien B menos el precio anterior, luego la dividimos por el precio inicial y multiplicamos por cien:

\[
\frac{\Delta p_B}{p_B} = \frac{p_{B1} - p_{B0}}{p_{B0}} \times 100
\]

Lo importante en esta instancia es tener presente que se trata de la variación del precio de otro bien, distinto del que utilizamos para calcular la variación de la cantidad demandada. Es por ejemplo el caso: si analizamos cómo afecta a la demanda de \textit{pamelos} la variación en el precio las \textit{naranjas}. Es decir que en este caso el precio no se comportará predeciblemente en forma inversa a la variación mencionada, como ocurre en el caso de la elasticidad-precio.

\textbf{III.} Por lo dicho en el párrafo anterior, entendemos que el resultado de la elasticidad cruzada puede ser positivo o negativo.

Aquí nos interesa justamente observar el signo que toma ese resultado. Para el cálculo de la elasticidad cruzada tenemos:

\[
E_c = \frac{\Delta q_A / q_A}{\Delta p_B / p_B}
\]

A modo de ejemplo, supongamos que el precio de un bien X aumenta 15\% y la cantidad demandada de un bien Z aumenta 5\%, como consecuencia de la variación enunciada:

\[
E_c = \frac{5\%}{15\%} = 0,33
\]

Si la cantidad demandada de un bien $\alpha$ disminuye 20\% frente al aumento del precio del bien $\beta$ de 10\%, entonces tendremos:

\[
E_c = \frac{-20\%}{10\%} = -2
\]

\subsection{Interpretación de los resultados}

Como se ha mencionado, la elasticidad cruzada puede ser positiva o negativa y justamente ese signo será el que indique el tipo de relación que existe entre ambos. Analicemos entonces qué representan tales valores.

\begin{itemize}
\item La elasticidad cruzada es mayor que cero ($E_c > 0$) cuando la relación entre las variables es directa; es decir que si el precio de un bien sube, la cantidad demandada del otro bien aumenta y si el precio del primero baja, la cantidad demandada del segundo disminuye. Cuando esto ocurre decimos que los bienes son \textbf{sustitutivos}, es decir que son alternativos en su uso. Si aumenta el precio de uno de ellos, la carne de vaca, dejamos de comprarlo y tomamos la alternativa que representa el otro bien, la carne de pollo, y por consiguiente, aumenta la demanda de aquel sustituto en este caso, la carne de pollo.

\item Por el contrario, la elasticidad cruzada es menor que cero ($E_c < 0$) cuando la relación entre las variables es inversa; es decir que si el precio de un bien sube, la cantidad demandada del otro bien disminuye y si el precio del primero baja, la cantidad demandada del segundo aumenta. Cuando esto ocurre decimos que los bienes son \textbf{complementarios}, es decir que el consumo de uno se ve acompañado por el consumo del otro. Si aumenta el precio de uno de ellos, por ejemplo una impresora, dejamos de comprarla y también dejamos de comprar el bien que consumimos conjuntamente con aquella: los cartuchos de tinta para la impresora.

\item Otro resultado posible es que la elasticidad cruzada sea cero. ¿Cuándo ocurre esto? Cuando la variación en el precio de un bien no produce una modificación en la cantidad demandada de otro, es decir que el numerador será cero. Decimos entonces que estos bienes no están relacionados, por lo tanto se denominan \textbf{bienes independientes}. Por ejemplo al subir el precio de los zapatos, la demanda de automóviles no se ve afectada, ni sube ni baja.
\end{itemize}

\section{Elasticidad precio de la oferta}

Cuando estudiamos la oferta de un bien, señalamos que los empresarios ofrecen más cantidades cuando sube el precio del bien y reducen las cantidades ofrecidas cuando baja el precio. Recuerden la pendiente positiva (función creciente) de la curva de oferta.

Nos interesa analizar cómo reaccionan las cantidades ofrecidas de un bien, para lo cual se utilizará el concepto de elasticidad.

\[
E_p \text{ de la oferta} = \frac{\text{Variación porcentual de la cantidad ofrecida}}{\text{Variación porcentual del precio}}
\]

\subsection{Aspectos matemáticos}

En la fórmula recién señalada, es posible observar que no hay signo negativo delante de la misma, ya que no es necesario porque si el precio sube, la cantidad ofrecida sube (las dos variaciones serán positivas y el resultado será positivo) o bien, si el precio baja la cantidad ofrecida disminuye (las dos variaciones serán negativas y el resultado será también positivo).

Por ejemplo: Si el precio del litro de la leche sube de \$2,85 a \$3,15, la cantidad que los productores lecheros llevarán al mercado en un mes pasará de 9.000 a 11.000 litros.

\[
\frac{\Delta q}{q} = \frac{11000 - 9000}{9000} \times 100 = 22,22\%
\]
\[
\frac{\Delta p}{p} = \frac{3,15 - 2,85}{2,85} \times 100 = 10,53\%
\]
\[
E_p = \frac{22,22\%}{10,53\%} = 2,11
\]

\subsection{Interpretación de los resultados}

\textbf{I.} Cuando la $E_p$ de la oferta es un número mayor a 1, decimos que la oferta es elástica o bien, que la cantidad ofrecida es muy sensible a la variación del precio. El numerador de la fórmula es mayor que el denominador.

\textbf{II.} Cuando la $E_p$ de la oferta es un número menor a 1, decimos que la oferta es inelástica o bien, que la cantidad ofrecida es poco sensible a las variaciones del precio. En este caso el numerador de la fórmula es menor que el denominador.

\textbf{III.} Cuando la $E_p$ de la oferta es igual a 1, decimos que la oferta tiene elasticidad unitaria o bien, que la cantidad ofrecida responde en la misma proporción a la variación del precio. El numerador es igual al denominador.

\textbf{IV.} Cuando la $E_p$ de la oferta es igual a 0, la oferta es perfectamente inelástica, ya que la cantidad ofrecida no cambia cuando cambia el precio. La curva de oferta es vertical.

\textbf{V.} Cuando la $E_p$ de la oferta tiende a infinito, decimos que es perfectamente elástica o bien, que frente un precio determinado, el oferente está dispuesto a llevar al mercado cualquier cantidad del producto. A un precio distinto, no existirá oferta.

El valor que toma la elasticidad de la oferta depende:

\begin{itemize}
\item De la capacidad de reacción de los empresarios frente a los cambios del precio. Dicha capacidad está ligada a las características propias del proceso productivo, que permitirá aumentar la producción en forma más o menos rápida, dependiendo de la necesidad o no de incorporar factores productivos.

\item Del plazo de tiempo considerado para el análisis. En el muy corto plazo la oferta de casi todos los productos tiende a ser rígida o inelástica, por ello, cualquier cambio de la demanda se traducirá en un aumento inmediato del precio. A medida que el plazo considerado es mayor la oferta se toma más elástica ya que es posible ajustar las cantidades ofrecidas como respuesta al cambio del precio.
\end{itemize}

\section{Aplicación del concepto de elasticidad precio de la demanda y de la oferta}

\subsection{Incidencia de un impuesto}

Cuando el gobierno decide gravar la venta de un bien con un impuesto, el precio que pagan los consumidores aumenta y el precio que reciben los productores disminuye. La magnitud de la variación de los precios dependerá de las elasticidades de la oferta y la demanda.

Si la demanda es más inelástica que la oferta, la mayor parte del impuesto recaerá sobre los consumidores. Si la oferta es más inelástica que la demanda, la mayor parte del impuesto recaerá sobre los productores.

\end{document}
